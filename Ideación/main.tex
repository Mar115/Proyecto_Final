\documentclass{article}
\usepackage[utf8]{inputenc}
\usepackage[spanish]{babel}
\usepackage{listings}
\usepackage{graphicx}
\graphicspath{ {images/} }
\usepackage{cite}

\begin{document}

\begin{titlepage}
    \begin{center}
        \vspace*{1cm}
            
        \Huge
        \textbf{Proyecto final}
            
        \vspace{0.5cm}
        \LARGE
            
            
        \vspace{1.5cm}
        
        \textbf{Marcela Florez} 
        
        \textbf{Ferney Mejía Pérez}
            
        \vfill
            
        \vspace{0.8cm}
            
        \Large
        Departamento de Ingeniería Electrónica y Telecomunicaciones\\
        Universidad de Antioquia\\
        Medellín-Antioquia\\
        Marzo de 2021
            
    \end{center}
\end{titlepage}

\tableofcontents
\newpage
\section{Sección introductoria}\label{intro}
El presente proyecto hace énfasis en la implementación de las ideas principales del proyecto final correspondiente al curso en proceso, informática II, la cuál tiene un enfoque de enseñanza en el lenguaje de programación C++, con el cual, se realizarán diversos ejercicios de aprendizaje con el fin de adquirir lógica computacional y la habilidad necesaria para dicho la solución de problemas comunes en la programación. Por lo tanto, para este proyecto se tiene de inicio la creación del esquema organizacional del cual se desprenderán las diferentes ideas que componen el cuerpo del juego, que corresponde a dicho proyecto final.

Ahora bien, con respecto al juego, se tiene una
\section{Sección de contenido} \label{contenido}
A continuación se presentan las ideas pertenecientes al esquema organizacional del juego:

\subsection{Entorno}
El juego tendrá un entorno desarrollado en el espacio sideral, en el cuál habrán múltiples galaxias a las cuales acceder
\subsection{Protagonistas}

\subsubsection{Antagonistas}

\subsection{Acciones}

\subsection{Recursos}

\subsection{Eliminación}

\subsection{Interacción}

\subsection{Finalidad}
Recolección de galaxias, planetas, estrellas
\newpage
\section{Referencias}
\bibliographystyle{IEEEtran}
\bibliography{references}
\cite{calistenia}


\end{document}
